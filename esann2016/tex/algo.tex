Neural Fitted Actor-Critic (NFAC) is a novel actor-critic algorithm whose roots in both CACLA and Q-Fitted Iteration.
It performs in an offline way by following 2 steps : 1) sample and collect data $(s_t, u_t, a_t, r_{t+1}, s_{t+1})$ into $\mathcal{D}_\pi$
from the environment given the current policy $\pi$, 2) try to improve the critic approximation and actor policy with $\mathcal{D}_\pi$.

% During the interaction phase, the actor selects the action $u_t \in A$  which correspond to the exploitation action (the output of the neural network)
% and performs the action $a_t$. The policy exploration can take different forms : Gaussian $a_t \sim \mathcal{N}(u_t, \sigma)$, 
% $\epsilon$-greedy 
% $a_t = 
% \begin{dcases} u_t, & \text{if }p > \epsilon\ |\ p \sim U(0,1) \\
%       \text{random}, & \text{otherwise}
% \end{dcases}$, etc.

\subsection{Critic}

The critic's update relies on several iterations as done in FQI. 
At each iteration $k$ a database $S \times \mathbb{R}$ is constructed knowing $V_k$ and $\mathcal{D}_\pi$.
For each sample of $D_\pi$ the input of the model is $s_t$ and the target to approximate is $r_{t+1} + \gamma V_k(s_{t+1})$.
If $s_{t+1}$ is a goal state or absorbing state, the target becomes only $r_{t+1}$.
\begin{equation}
 V_{k+1} = \underset{V \in \mathcal{F}_c}{\text{argmin}} 
 \sum_{ (s_t, r_{t+1}, s_{t+1}) \in \mathcal{D}_\pi}
%  \frac{1}{\hat{\mu}(s_t)} \frac{\pi(a|s)}{\pi_b(a|s)}
 \Big[ V(s_t) - \big( r_{t+1} + \gamma V_k(s_{t+1}) \big) \Big]^2
\end{equation}
$\mathcal{F}_c$ is the search space of the critic ; a subset of learnable functions
(for instance a multi-layer percetron with a fixed number of hidden units and activation function).
Multiple databases are required since the target to learn for $V_{k+1}$ is dependent on $V_k$,
it represents an iteration of {\em value iteration}.

As the best next action cannot be known easily, the next state and V depends of $\pi$ : NFAC is on-policy unlike FQI.
% Unlike FQI, NFAC is on-policy since the optimization to find the best action is not trivial 
% with continuous environment. 
Thus it is not necessary to learn the full action-value function $Q$
where $V$ is enough and easier to approximate. This also implies that $\mathcal{D}_\pi$ is only
used for one episode as the policy $\pi$ is changing.

\subsection{Actor}

% It is computionnaly unviable to try to optimize a non-linear value function to find
% the best action, so 
The actor update
relies on the temporal difference error $\delta = \big( r_{t+1} + \gamma V_k(s_{t+1}) \big) - V(s_t)$.
Since $\delta$ depends on the approximation of $V$, the actor update is accomplished before the critic update.
Like CACLA algorithm, the update is performed towards a direction only when $\delta > 0$.
However the process is not online but done according to $\mathcal{D}_\pi$.
A unique database $ S \times A$ is enough : there is no incremental process as with the critic
because there isn't a self-dependency in the update formula :

\begin{equation}
 \hat{\pi}^* = \underset{\pi \in \mathcal{F}_a}{\text{argmin}} 
 \sum_{ (s_t,a_t, a_t) \in \mathcal{D}_\pi}
 \begin{dcases}
  \big( \pi(s_t) - a_t \big)^2 ,& \text{if } \delta > 0\\
  \big( \pi(s_t) - u_t \big)^2 ,& \text{otherwise}
 \end{dcases}
\end{equation}

Notice that an exploitation case, where $a_t$ is equals to $u_t$, could also
result in a policy update. The main difference with CACLA is when an action leads to an unexpected higher value-function,
this action is not only reinforced, but the policy tries to reproduce it.


From a technical point of view, Rprop back-propagation algorithm \cite{Igel2000, Riedmiller1992} is used in both update.
No momentum is defined since the value function needs to be flexible.
