\subsection{Reinforcement learning}

RL is a framework that models sequential decision
problems where an agent learns to take better decisions while interacting with
its environment. After an agent performs an action, the state changes and the
agent receives a numerical value, which can be null, called reward, that encodes information
about the quality of the state just reached. The goal of the agent is to maximize
its long-term expected total reward. Because it is possible that actions associated 
with low reward will allow the agent to reach high-reward states, the agent
needs to estimate somehow (by trial-and-error) the value of taking an action in
a given state.

The underlying formalism of RL is that of {\em Markov Decision Processes} (MDPs).
An MDP is formally defined as a tuple $\langle S, A, T, R \rangle$ where $S$ is a set of states, $A$ a set of actions, $T : S \times A \times S \rightarrow [0, 1]$ 
are transition probabilities between states
($T(s,a,s')=p(s'|a,s)$ is the probability of reaching state $s'$ from state $s$ after executing action $a$) and 
$R : S \times A \rightarrow \mathbb{R}$ is a reward signal. A \textit{policy} $\pi : S \times A \rightarrow [0, 1]$ encodes how the agent will behave (the probability
to take an action in a given state).
The {\em optimal} policy $\pi^*$ is one of the policy maximizing the expected discounted reward, that is :
\begin{equation}
\pi^* = \underset{\pi}{\operatorname{arg\,max}}\ J(\pi) = \underset{\pi}{\operatorname{arg\,max}}\ \mathbb{E} \Big[ \sum\limits_{t=0}^{\infty} \gamma^t \times R(s_t, \pi_t(s_t)) \Big]
\end{equation}
where $t$ denotes a time step and $0 < \gamma < 1$ is a discount factor.
In MDPs, since $T$ and $R$ are given, an optimal policy $\pi^*$ can be computed offline using dynamic programming (e.g., value iteration, policy                                                                  
iteration) \cite{Howard1960} for instance.

In RL setting, the agent however does not know the transition probabilities $T$ and the reward function $R$.
There are two main approaches for learning an optimal policy in RL: model-based approach and model-free approach.
While {\em model-based} methods aim at learning $T$ and $R$, which is usually difficult in continuous environment,
{\em model-free methods} are computationally more viable techniques
that perform online learning by storing an estimation of $Q : S \times A \rightarrow \mathbb{R}$, the (sequential) values of actions in each state,
or $V : S \rightarrow \mathbb{R}$ the value of each state.

\begin{equation}
  Q^\pi(s, a) = \mathbb{E}_\pi \Big[ \sum_{k=0}^{\infty} \gamma^k r_{t+k+1} \Big| s_t = s, a_t = a \Big]
\end{equation}

In a small and discrete environment it is possible to store Q or V as a table.
Notice however that in many tasks, the number of states and actions are
too large so that the agent cannot store it in a table. The problem of
continuous states can be tackled with discretization; however the number of
states will often explode. Even with the memory and computational power to
handle very large tables, reinforcement learning with Q values represented in the
tabular form suffers from the lack of generalization, meaning that what is learned
in a state does not impact the knowledge about the value of similar states.
A solution to these problems is to use an approximation function, that provide
a means for both efficient computation and support for generalization.


% \begin{equation}
%  V^\pi(s) = \mathbb{E}_\pi \Big[ \sum_{k=0}^{\infty} \gamma^k r_{t+k+1} \Big| s_t = s \Big]
% \end{equation}
% 



\subsection{Critic only}
The main feature of these methods is to derive a policy directly from the value function.
\begin{equation}
 \pi(s) = M(Q, s, A) = \underset{a \in A}{\operatorname{arg\,max\ }} Q(s,a)
\end{equation}
where the function $M$ can be the $\operatorname{arg\,max}$ operator, Boltzmann soft max, etc.
They have been widely studied since they works well in discrete case.
Thus the whole problem is how to learn those value functions.
Several algorithms have been designed with different update rules : QLearning, Sarsa, LSTD, etc.
Fitted Q Iteration (FQI) \cite{Riedmiller2005} uses non-linear neural networks with a good data efficiency.
This is why it will be more detailed here.
It can be seen as a value iteration based on previous collected data $(s_t, a_t,r_{t+1}, s_{t+1})_{t=\{1,...,N\}}$ of one episode containing $N$ steps.
The Q function is incrementally learned by : 
\begin{equation}
 Q_{k+1} = \underset{Q \in \mathcal{F}}{\text{arg min}} \sum_{t=1}^{N} \Big[ Q(s_t, a_t) - \big( r_{t+1} + \gamma \underset{a' \in A}{\text{ max }} Q_k(s_{t+1}, a') \big) \Big]^2
\end{equation}

% \begin{equation}
%  Q(s_t, a_t) =  r_{t+1} + \gamma \underset{a' \in A}{\text{ max }} Q_k(s_{t+1}, a')
% \end{equation}

where $\mathcal{F}$ is a subset of multilayer perceptron with a fixed number of hidden neurons.
Both the state and the action are inputs of the neural network and the output is a scalar value
approximating $\hat{Q}(s,a)$. Another big advantage of FQI is that it can use
the Rprop \cite{Igel2000} backpropagation algorithm, thus it doesn't need a learning rate parameter
to be tuned.

% \begin{equation}
%  \Delta W = - \eta \big( Q(s_t, a_t) - r_{t+1} + \gamma Q_k(s_{t+1}, \pi(s_{t+1}) \big) \frac{\partial Q(s_t, a_t)}{\partial W}
% \end{equation}
The main issue with critic only methods is that they are not applicable to continuous actions since
the argmax operator needs to enumerate all possible actions ; which cannot be done in infinite space.

\subsection{Actor only}

On the other side, it is possible to define a parametric policy $\pi_{\theta}$ without value function \cite{Sutton1999}.
The goal is to found the best parameters according to the cost function $J(\pi_{\theta})$. 
It is no longer the space $S \times A$ that is explored but a finite-dimensional parameter space $\Theta$ ;
which is mapped to the space of policies. It is also called direct policy search.
Each parameter is updated according to his contribution to the cost function 
$ \frac{\partial J}{\partial \theta} $ \cite{Kober2010a}.
Genetic algorithms like CMAS-ES \cite{Hansen2001} can also enter is this category even if 
they do not necessarily use gradient.

The major drawback is the high variability of the cost J which can result in poor policies.

\subsection{Actor Critic}

Those algorithms \cite{Konda2003} tries to combine the advantage of both previous methods.
The critic learns the value function to reduce the variability of the approximation of J 
and the actor learns the policy allowing continuous actions.
Cacla \cite{VanHasselt2007a} is capable to use neural networks for both the critic and the actor.
Temporal difference error $\delta = r_t + \gamma V(s_{t+1}) - V(s_t)$ is used by the critic update :
\begin{equation}
 V_{t+1}(s) = V_t(s) + \alpha_v \delta
\end{equation}
As for the actor policy, parameterized by $\theta$ :
\begin{equation}
 \theta_{t+1} = \theta_t + 
 \begin{cases}
%  \underset{\delta > 0}{\text{1}} 
 \alpha_a (\widetilde{a}_t - Ac(s_t))
 \frac{\partial Ac(s_t)}{\partial \theta_t} , & \text{if } \delta > 0\\ \\
 0, & \text{otherwise}
 \end{cases}
\end{equation}
where $Ac(s_t)$ is the selected action in the exploitation case and $\widetilde{a}_t$ the performed action.
Cacla is a online algorithm that doesn't retains data from previous experiences. %except through the value function. 

% On the contrary, fitted natural actor critic \cite{Melo2008} tries to exploit 
% at the maximum the data already collected by replaying them with importance sampling. 
% The critic updates is close the Q Fitted Iteration :
% \begin{equation}
%  V_{k+1} = \underset{V \in \mathcal{F}}{\text{arg min}} \sum_{t=1}^{N} 
%  \frac{1}{\hat{\mu}(s_t)} 
%  \frac{\pi(a_t|s_t)}{\pi_0(a_t|s_t)}  
%  \Big[ V(s_t) - \big( r_{t+1} + \gamma V_k(s_{t+1}) \big) \Big]^2
% \end{equation}
% where $\pi_0$ is the policy that have generated this data and $\hat{\mu}$ the empirical 
% probability distribution over $S$. The actor update follows natural gradient descent. It won't
% be discussed here as it relies on linear approximation.

\subsection{Non Linear Hypothesis}

When the space A or S are continuous, classical solutions try to learn to approximate value function 
using a linear combination of provided basis functions. 
These basis function must be define by an expert ; their nature and number limits the expressivity 
of the learned function. It also impedes the agent's capacity to develop its own representations.
Therefore, following hypothesis 2), the use of non-linear functions (like neural networks) 
will be explored as they have a greater expressiveness range and avoid the definition of basis functions
{\it a priori}. Moreover no model of the environment nor prior trajectories should be provided to the algorithm.


% Most of the current algorithms define a very small learnable function domain, usually a linear approximation
% of provided basis functions. They must therefore be selected by an expert of the domain. It also restricts 
% the capacity of the agent to develop its own representations. % (which could be better than a human).
% Thus, the second hypothesis mainly implies to use non-linear model (as neural networks) to avoid the definition of basis functions and
% making possible the emergence of internal representations in a bigger domain of learnable functions.
% Moreover no model of the environment nor prior trajectories should be provided to the algorithm.

% Thus the data produced should be well exploited and don't be forgotten just after being used.


% Several state of the art algorithms that fulfill 2 different hypothesis have been presented :
% 
% \begin{tabular}{|p{3.4cm}|>{\centering\arraybackslash}p{1.2cm}|>{\centering\arraybackslash}p{3cm}|>{\centering\arraybackslash}p{1.8cm}|}
%  \hline
%  ~\mbox{Algorithms} ~ \mbox{Criterions} & Cacla, CMAES & Fitted Natural Actor-critic, Power & Fitted Q iteration \\ \hline
%  1) Continuous Actions & $\times$ & $\times$ & \\ \hline
%  2) Low Knowledge & $\times$ & & $\times$ \\ \hline
%  3) Data Efficient &  & $\times$ & $\times$\\ \hline
% \end{tabular}
% 

